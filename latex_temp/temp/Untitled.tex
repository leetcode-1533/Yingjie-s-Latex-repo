\documentclass[11pt]{article}

\begin{document}

\begin{eqnarray*}
h(z)&=&\frac{3}{1+z^{-1}-2z^{-2}}\\
&=&\frac{3}{(1+2z^{-1})(1-z^{-1})}
\end{eqnarray*}
Applying the decomposition formula:
\begin{eqnarray*}
h(z)&=&\frac{3}{(1+2z^{-1})(1-z^{-1})}\\
&=&\frac{2}{1+2z^{-1}}+\frac{1}{1-z^{-1}}
\end{eqnarray*}


Because the system is a casual system, we then can apply the inverse z-transform table \\

Refer to the table:
for
 $$\frac{1}{1+2z^{-1}}$$

 the related h[n] should be:
 $$(-2)^nu[n]$$
 
for $$\frac{1}{1-z^{-1}}$$

the related h[n] should be:
$$u[n]$$\\

Finally, according to linear property:
\begin{eqnarray*}
h(z)&=&\frac{3}{1+z^{-1}-2z^{-2}}\\
h(z)&=&\frac{3}{(1+2z^{-1})(1-z^{-1})}\\
h(z)&=&\frac{2}{1+2z^{-1}}+\frac{1}{1-z^{-1}}\\
h[n]&=&2*(-2)^nu[n]+u[n]
\end{eqnarray*}









\end{document}
